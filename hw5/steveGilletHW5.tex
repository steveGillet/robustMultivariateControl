\documentclass{article}
\usepackage[margin=1in]{geometry} % For setting page margins
\usepackage{amsmath}
\usepackage{amssymb} % For math symbols and equations
\usepackage{graphicx} % For including images
\usepackage{hyperref} 
\usepackage{enumitem}
\usepackage{float}
\usepackage{listings}
\usepackage{xcolor}
\usepackage{caption}

\renewcommand{\thesection}{\arabic{section}.}
\renewcommand{\thesubsection}{(\alph{subsection})}

\lstdefinestyle{matlabstyle}{
    language=Matlab,              % Specify the language
    basicstyle=\ttfamily\footnotesize\color{black}, % Code font
    keywordstyle=\color{blue}\bfseries, % Keywords in blue
    stringstyle=\color{orange},    % Strings in green
    commentstyle=\color{magenta}, % Comments in magenta
    numbers=left,                 % Line numbers on the left
    numberstyle=\tiny\color{black},% Line number style
    stepnumber=1,                 % Line number increment
    breaklines=true,              % Line breaking
    frame=single,                 % Border around code
    backgroundcolor=\color{white},
    tabsize=4,                    % Tab size
    showstringspaces=false,       % Don't show spaces in strings
}

\begin{document}

\title{
    \begin{tabular}{@{}l@{}}
        \textbf{Class:} Robust Multivariate Control \\
        \textbf{Professor:} Dr. Sean Humbert \\
        \textbf{TAs:} Santosh Chaganti \\
        \textbf{Student:} Steve Gillet \\
        \textbf{Date:} \today \\
        \textbf{Assignment:} Homework 5
    \end{tabular}
}

\author{}
\date{}

\maketitle

\section{}

\textit{For the simple SISO negative feedback system with \(P(s) = \frac{1}{s-1}\) and \(C(s) = \frac{s-1}{s+2}\), show that at least one transfer function from exogenous signals \(r\) and \(d_I\) to the internal signals \(e\), \(u\) and \(y\) is unstable due to the right-half plane pole/zero cancellation of \(s - 1\) in the loop transfer function \(L = PC\).}

\begin{figure}[H]
    \centering
    \includegraphics[width=\textwidth]{p1ryDiagram.jpg}
\end{figure}

I will start with the transfer function from \(r\) to \(y\) which for this SISO unity feedback system is:
\[
\frac{PC}{1 + PC}
\]

Or:

\[
\frac{(\frac{1}{s-1})(\frac{s-1}{s+2})}{1 + (\frac{1}{s-1})(\frac{s-1}{s+2})} = \frac{\frac{1}{s+2}}{1+\frac{1}{s+2}} = \frac{\frac{1}{s+2}}{\frac{s+2}{s+2}+\frac{1}{s+2}} = \frac{\frac{1}{s+2}}{\frac{s+3}{s+2}} = \frac{1}{s+3}
\]

This transfer function is not only stable but it was made stable by the addition of the controller and the pole/zero cancellation. Let us see some other signals, eh.

Full system diagram with $d_I$:

\begin{figure}[H]
    \centering
    \includegraphics[width=\textwidth]{p1diyDiagram.jpg}
\end{figure}

Let's try r$\to$u:

Direct path: \(C\)
Return path: \(1 + PC\)

\[
\frac{C}{1+PC}
\]

\[
\frac{\frac{s-1}{s+2}}{1 + \frac{1}{s-1}\frac{s-1}{s+2}} =
\]

\[
\frac{\frac{s-1}{s+2}}{1 + \frac{1}{s+2}} = \frac{\frac{s-1}{s+2}}{\frac{s+3}{s+2}} = \frac{s-1}{s+3}
\]

\[
= \frac{s-1}{s+3}
\]

Again the pole is at -3 and the transfer function is stable.
Let's try $d_I \rightarrow y$:

\[
\frac{P}{1 + CP}
\]

\[
\frac{\frac{1}{s-1}}{\frac{s+3}{s+2}}
\]

\[
= \frac{s+2}{(s-1)(s+3)}
\]

There we go, a pole at $s = 1$, the transfer function from $d_I$ to y is unstable.

\section{}

\textit{For the following longitudinal model for an F-4 Phantom with 4 perturbation states (pitch rate \(\Delta q\), forward speed \(\Delta u\), angle of attack \(\Delta \alpha\), pitch angle \(\Delta \theta\)), 2 inputs (elevator deflection \(\Delta \delta_e\), flaperon deflection \(\Delta \delta_f\)) and 2 outputs (\(\Delta q\) and \(\Delta \alpha\)). Note that the \(D\) matrix entries are zeros for this output selection.}

\[
A = \begin{pmatrix}
-0.8 & -0.0006 & -12 & 0 \\
0 & -0.014 & -16.64 & -32.2 \\
1 & -0.0001 & -1.5 & 0 \\
1 & 0 & 0 & 0
\end{pmatrix}, \quad
B = \begin{pmatrix}
-19 & -2.5 \\
-0.66 & -0.5 \\
-0.16 & -0.6 \\
0 & 0
\end{pmatrix}.
\]

\subsection{}
\textit{Generate a MATLAB plot that contains the singular value bode plots in red combined with the bode magnitude plots of the 4 individual transfer functions from the 2 inputs to 2 outputs. Does the maximum singular value plot envelope the individual transfer function plots?}

I generated the system, transfer function, and plots using the 'ss', 'ss2tf', and 'bode' Matlab functions.
You can see my implementation below:

\begin{lstlisting}[style=matlabstyle]
A = [ -0.8 -0.0006 -12 0; 0 -0.014 -16.64 -32.2; 1 -0.0001 -1.5 0; 1 0 0 0];
B = [ -19 -2.5; -0.66 -0.5; -0.16 -0.6; 0 0];
C = [1 0 0 0; 0 0 1 0];
D = [0 0; 0 0];

phantomSS = ss(A,B,C,D);
[sv,wout] = sigma(phantomSS, {0,10});

figure;
plot(wout, sv, 'Color', [0.545, 0, 0]);
grid on;
title('Singular Value Bode Plot');
xlabel('Frequency [rad/s]');
ylabel('Absolute Gain [absolute units]');
hold on;

[num, den] = ss2tf(A,B,C,D,1);
tf11 = tf(num(1,:), den);
tf12 = tf(num(2,:), den);
[num, den] = ss2tf(A,B,C,D,2);
tf21 = tf(num(1,:), den);
tf22 = tf(num(2,:), den);

[mag11, phase11, wout11] = bode(tf11, {0,10});
[mag12, phase12, wout12] = bode(tf12, {0,10});
[mag21, phase21, wout21] = bode(tf21, {0,10});
[mag22, phase22, wout22] = bode(tf22, {0,10});
plot(wout11, squeeze(mag11), 'Color', [1,0.65,0]);
plot(wout12, squeeze(mag12), 'Color', [1,0.65,0]);
plot(wout21, squeeze(mag21), 'Color', [1,0.65,0]);
plot(wout22, squeeze(mag22), 'Color', [1,0.65,0]);
legend('System Singular Values', 'Individual Transfer Function Magnitudes');    
\end{lstlisting}

The plot is below and you can see that yes in fact thee maximum singular value plot (in red) envelopes the individual transfer functions (in orange):

\begin{figure}[H]
    \centering
    \includegraphics[width=\textwidth]{singularValuePlot.png}
\end{figure}

\end{document}