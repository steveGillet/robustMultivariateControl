\documentclass[12pt, letterpaper]{article}

% Packages
\usepackage[margin=1in]{geometry} % For setting page margins
\usepackage{amsmath, amssymb} % For math symbols and equations
\usepackage{graphicx} % For including images
\usepackage{hyperref} % For hyperlinks
\usepackage{listings}
\usepackage{enumitem}
\usepackage{float}
\usepackage{listings}
\usepackage{xcolor}
\usepackage{caption}

\renewcommand{\thesection}{\arabic{section}.}

\lstdefinestyle{matlabstyle}{
    language=Matlab,              % Specify the language
    basicstyle=\ttfamily\footnotesize\color{black}, % Code font
    keywordstyle=\color{blue}\bfseries, % Keywords in blue
    stringstyle=\color{orange},    % Strings in green
    commentstyle=\color{magenta}, % Comments in magenta
    numbers=left,                 % Line numbers on the left
    numberstyle=\tiny\color{black},% Line number style
    stepnumber=1,                 % Line number increment
    breaklines=true,              % Line breaking
    frame=single,                 % Border around code
    backgroundcolor=\color{white},
    tabsize=4,                    % Tab size
    showstringspaces=false,       % Don't show spaces in strings
}

\begin{document}

\title{
    \begin{tabular}{@{}l@{}}
        \textbf{Class:} Robust Multivariate Control \\
        \textbf{Professor:} Dr. Sean Humbert \\
        \textbf{TAs:} Santosh Chaganti \\
        \textbf{Student:} Steve Gillet \\
        \textbf{Date:} \today \\
        \textbf{Assignment:} Homework 3
    \end{tabular}
}

\author{}
\date{}

\maketitle

\section{D-Stability of a System}
\textit{
    "In lecture it was shown that the LMI to verify stability of the system $\dot{x} = Ax$ is given by:
}
\[
\begin{cases}
P > 0 \\
A^T P + P A < 0,
\end{cases}
\]
\textit{
    which verifies that the eigenvalues of $A$ are in the left half plane. Modifications can be made to this LMI that provide conditions to verify the eigenvalues of $A$ are in a subset $\mathbb{D}$ of the left half plane, which is called $\mathbb{D}$-stability. The simplest case is to verify $\lambda_i(A)$ are in region that lies to the left of some value of $\alpha$ in the left half plane. In this case the resulting LMI conditions are:
}
\[
\begin{cases}
P > 0 \\
A^T P + P A + 2\alpha P < 0.
\end{cases}
\]
\textit{
    Code up this new LMI as a \texttt{feasp} problem using the MATLAB LMI Toolbox and use it to estimate the largest region of $\mathbb{D}$-stability for the following systems $\dot{x} = Ax$ where
}
\[
A = \begin{pmatrix}
0 & 1 \\
-1 & -2
\end{pmatrix}, \quad
A = \begin{pmatrix}
-1.5 & 1 & 0.1 \\
-4 & -1 & 0 \\
0 & 1 & 0
\end{pmatrix}.
\]
\textit{"}

I used the following code to set up the feasability problem.

\begin{lstlisting}[style=matlabstyle]
A = [-1.5 1 0.1; -4 -1 0; 0 1 0];
% A = [0 1; -1 -2];
nStates = size(A, 1);
disp(eig(A));

setlmis([])

P = lmivar(1, [nStates 1]);
alpha = 0.05;

lmiterm([-1 1 1 P],1,1);
lmiterm([2 1 1 P],1,A,'s');
lmiterm([2 1 1 P],2*alpha,1);

lmiDstab = getlmis;

[tmin, pfeas] = feasp(lmiDstab);
P = dec2mat(lmiDstab,pfeas,P);
disp(P);    
\end{lstlisting}

I set alpha to .99 here.
I confirmed alpha by printing out the Eigenvalues of A using the 'eig' function and then kind of played around with different values of alpha around the Eigenvalues.
The output for the first A matrix is:

Eigenvalues:
-1
-1

Solver for LMI feasibility problems $L(x) < R(x)$

This solver minimizes  $t$  subject to  $L(x) < R(x) + t*I$

The best value of $t$ should be negative for feasibility

Iteration   :    Best value of t so far 

1                        0.067076

2                        0.010110

3                        0.010110

4                    2.180764e-03

5                    1.491047e-03

6                    1.491047e-03

7                    3.249250e-04

8                    -9.440735e-04

Result:  best value of t: -9.440735e-04

f-radius saturation:  0.000\% of R =  1.00e+09

1.0e+03 * \\


The output for the second A matrix  with alpha set to 0.05 is:

Eigenvalues:

-1.2124 + 1.9616i

-1.2124 - 1.9616i

-0.0752 + 0.0000i 

Solver for LMI feasibility problems $L(x) < R(x)$

This solver minimizes  $t$  subject to  $L(x) < R(x) + t*I$

The best value of $t$ should be negative for feasibility

Iteration   :    Best value of t so far 

1                        0.060635

2                       -0.027178

Result:  best value of t:    -0.027178

f-radius saturation:  0.000\% of R =  1.00e+09

\section{Spectral Norm of a Matrix}

\textit{
    "In lecture we derived the following LMI representing the inequality $\|A\|_2 = \sigma(A) < \gamma$,
}

\[
\begin{pmatrix}
\gamma^2 I & A^T \\
A & I
\end{pmatrix} > 0.
\]
\textit{
Code up this LMI as a \texttt{mincx} problem using the MATLAB LMI Toolbox. Assume the minimization variable is $\rho = \gamma^2$:
}
\[
\begin{cases}
\min \rho \\
\begin{pmatrix}
\rho I & A^T \\
A & I
\end{pmatrix} > 0.
\end{cases}
\]

\textit{
For the following matrices, use the code above to estimate $\sigma(A)$. Remember to back solve for $\gamma = \sqrt{\rho}$.
}

\[
A_1 = \begin{pmatrix}
2 & -1 \\
-1 & 2
\end{pmatrix}, \quad
A_2 = \begin{pmatrix}
-1 & 1 & 0 \\
1 & -1 & 1 \\
0 & 1 & 1
\end{pmatrix}.
\] \textit{"}

I used the following Matlab code:

\begin{lstlisting}[style=matlabstyle]
% A = [2 -1; -1 2];
A = [-1 1 0; 1 -1 1; 0 1 1];
disp(svd(A));

nStates = size(A,1);

setlmis([])

rho = lmivar(1, [1 0]);

lmiterm([-1 1 1 rho],1,1);
lmiterm([-1 1 2 0],A');
lmiterm([-1 2 2 0],1);

spectralLmi = getlmis;

c = mat2dec(spectralLmi,rho);

[copt, xopt] = mincx(spectralLmi, c);
rhoVal = dec2mat(spectralLmi, xopt, 1);
gamma = sqrt(rhoVal);
disp(gamma);    
\end{lstlisting}

I print out the singular values of the A matrix first so I can confirm my answers.
Output for first A matrix:

Singular Values: 

3.0000

1.0000

Solver for linear objective minimization under LMI constraints 

Iterations   :    Best objective value so far 
 
     1

     ***                 new lower bound:     0.472136

     2                   9.346101

     ***                 new lower bound:     2.184006

     3                   9.132359

     ***                 new lower bound:     6.560658

     4                   9.014443

     ***                 new lower bound:     7.678249

     5                   9.014443

     ***                 new lower bound:     8.294599

     6                   9.014443

     ***                 new lower bound:     8.754266

     7                   9.000200

     ***                 new lower bound:     8.914698

 Result:  feasible solution of required accuracy

 best objective value:     9.000200

 guaranteed relative accuracy:  9.50e-03

 f-radius saturation:  0.000\% of R =  1.00e+09 
 
Gamma: 

    3.0000 \\

Second A matrix output:

Singular Values: 

2.1701

1.4812

0.3111


 Solver for linear objective minimization under LMI constraints 

 Iterations   :    Best objective value so far 
 
     1

     ***                 new lower bound:    -0.208712

2                   4.883681

     ***                 new lower bound:     2.636568

3                   4.730055

     4                   4.730055

     ***                 new lower bound:     3.484965

5                   4.730055

     ***                 new lower bound:     4.359399

6                   4.713832

     ***                 new lower bound:     4.613672

7                   4.713832

     ***                 new lower bound:     4.693051

 Result:  feasible solution of required accuracy

 best objective value:     4.713832

          guaranteed relative accuracy:  4.41e-03

          f-radius saturation:  0.000\% of R =  1.00e+09 
 
Gamma: 

    2.1711

\section{$H_\infty$ Norm of a System}

\textit{"In this problem, you will use an alternate LMI form for the Bounded Real Lemma and implement it as a generalized eigenvalue problem (gevp) using the MATLAB LMI Toolbox. If we start with part (c) from the Bounded Real Lemma theorem from lecture and move the }$\gamma^2$\textit{ term to the right side, we have}

\[
\begin{pmatrix}
A^T P + PA + C^T C & PB + C^T D \\
B^T P + D^T C & D^T D
\end{pmatrix}
<
\gamma^2
\begin{pmatrix}
0 & 0 \\
0 & 1
\end{pmatrix}.
\]

\textit{In this form with } $P$ \textit{as the matrix variable, it can be cast as a generalized eigenvalue problem}

\[
\begin{cases}
\min \lambda \\
A(x) < \lambda B(x) \\
B(x) > 0 \\
C(x) > 0,
\end{cases}
\]

\textit{where } $\lambda = \gamma^2$, \textit{ } $A(x)$ \textit{ is the left hand side of the above expression, } $C(x) = P$, \textit{ and}

\[
B(x) = 
\begin{pmatrix}
\epsilon & 0 \\
0 & 1
\end{pmatrix},
\]

\textit{Here we need to add a small number } $\epsilon \approx 0.00001$ \textit{ to the matrix so that it is positive definite for numerical stability.}

\textit{(a) Implement the above LMI as a function that accepts matrices } $A$, $B$, $C$, \textit{ and } $D$ \textit{ from a general state space representation } $\dot{x} = Ax + Bu$, $y = Cx + Du$ \textit{ of a system } $G$ \textit{ and uses the function } \texttt{gevp} \textit{ to compute the infinity norm for the system, } $\|G\|_{\infty}$. \textit{ Be sure to back solve for } $\gamma = \sqrt{\lambda}$ \textit{ in your function.}

\textit{(b) Use your code from (a) to compute } $\|G\|_{\infty}$ \textit{ for the following stable (from Problem 1) dynamical systems. You can check your results using the MATLAB function } \texttt{hinfsyn(sys)} \textit{ where } \texttt{sys} \textit{ is a state space object constructed using the command } \texttt{ss}\textit{.}

\[
\begin{aligned}
\dot{x} &= \begin{pmatrix}
0 & 1 \\
-1 & -2
\end{pmatrix} x + \begin{pmatrix}
0 \\
1
\end{pmatrix} u, & 
\dot{x} &= \begin{pmatrix}
-1.5 & 1 & 0.1 \\
-4 & -1 & 0 \\
1 & 0 & 0
\end{pmatrix} x + \begin{pmatrix}
-0.2 \\
-1.8 \\
0
\end{pmatrix} u \\
y &= \begin{pmatrix}
1 & 0 \\
0 & 2
\end{pmatrix} x, & 
y &= \begin{pmatrix}
1 & 0 & 0 \\
0 & 1 & 0 \\
0 & 0 & 1
\end{pmatrix} x + \begin{pmatrix}
1 \\
0 \\
0
\end{pmatrix} u
\end{aligned}
\]
\textit{"}

This is how I set the problem up in Matlab:

\begin{lstlisting}[style=matlabstyle]
A = [-1.5 1 0.1; -4 -1 0; 1 0 0];
B = [-0.2; -1.8; 0];
C = [1 0 0; 0 1 0; 0 0 1];
D = [1; 0; 0];

% A = [0 1; -1 -2];
% B = [0; 1];
% C = [1 0; 0 2];
% D = [0];

nStates = size(A,1);
nOutputs = size(C,1);
nInputs = size(B,1);

setlmis([]);

P = lmivar(1, [nStates 1]);

epsilon = 1e-5;

lmiterm([-1 1 1 P], 1, 1);

lmiterm([-2 1 1 0], epsilon);
lmiterm([-2 2 2 0], 1);

lmiterm([3 1 1 P], A', 1, 's');
lmiterm([3 1 1 0], C'*C);
lmiterm([3 1 2 P], 1, B);
lmiterm([3 1 2 0], C'*D);
lmiterm([3 2 2 0], D'*D);
lmiterm([-3 1 1 0], epsilon);
lmiterm([-3 2 2 0], 1);

lmisys = getlmis;

[aa, xopt] = gevp(lmisys, 1);

disp(sqrt(aa));

sys = ss(A,B,C,D);
disp(hinfnorm(sys));    
\end{lstlisting}

I put the system into a state space model using the 'ss' Matlab function and then print out the 'hinfnorm' results to confirm my answers at the end.

And this was the result for the first system:

Solver for generalized eigenvalue minimization 

Iterations   :    Best objective value so far 

1                 309.375000

2                 146.808105

3                 101.297593

4                  69.895339

5                  48.227784

6                  33.277171

7                  22.961248

8                  15.843261

9                  10.931850

10                   7.542977

11                   5.204654

12                   2.246454

13                   2.246454

14                   2.246454

15                   2.223989

16                   2.201749

17                   1.498196

***                 new lower bound:    -0.333358

18                   1.440960

***                 new lower bound:    -0.276122

19                   1.440960

20                   1.387301

***                 new lower bound:     0.608383

21                   1.387301

22                   1.343377

***                 new lower bound:     1.005225

23                   1.338094

24                   1.338094

***                 new lower bound:     1.171659

25                   1.335493

***                 new lower bound:     1.297743

26                   1.333724

***                 new lower bound:     1.298904

27                   1.333724

Result:  feasible solution
        
best value of t:     1.333724
        
guaranteed relative accuracy:  2.61e-02

f-radius saturation:  0.000\% of R =  1.00e+08 

Termination due to SLOW PROGRESS:
        
the gen. eigenvalue t decreased by less than
        
1.000\% during the last 5 iterations.

Gamma: 
1.1549

'hinfnorm'
1.1547 \\

And this was the result for the second system:

Solver for generalized eigenvalue minimization 

Iterations   :    Best objective value so far 

1

    2

3

4

5

6

7

8

9

10

11

12

13

14

15

16

17

18

19

20

21

22

23

24

25

* switching to QR

* upper bound on optimal value set to  1.00e+08

26

* upper bound on optimal value set to  1.00e+11

27

28

Result:  infeasibility

Gamma: 

'hinfnorm': 

Inf

\end{document}